\begin{table}[!ht]
\centering
\resizebox{\textwidth}{!}{\begin{tabular}{lrrrrrrrrrrrrrrr}
\toprule
\textbf{Algorithm} & \multicolumn{5}{l}{Chase-Lev} & \multicolumn{5}{l}{Cilk THE} & \multicolumn{5}{l}{Idempotent LIFO} \\
\textbf{Operation} &       Puts &      Takes & Difference (\%) & Surplus (\%) & Executed Surplus (\%) &       Puts &      Takes & Difference (\%) & Surplus (\%) & Executed Surplus (\%) &            Puts &      Takes & Difference (\%) & Surplus (\%) & Executed Surplus (\%) \\
\textbf{Processes} & \multicolumn{4}{l}{} & \multicolumn{4}{l}{} & \multicolumn{4}{l}{}\\ \midrule
\textbf{1 } & 1000000.00 & 1000000.00 &           0.00 &        0.00 &                 0.00 & 1000000.00 & 1000000.00 &           0.00 &        0.00 &                 0.00 &      1000000.00 & 1000000.00 &           0.00 &        0.00 &                 0.00 \\
\textbf{8 } & 1025175.60 & 1000138.80 &           2.44 &        2.46 &                 0.01 & 1024845.60 & 1000102.40 &           2.41 &        2.42 &                 0.01 &      1026934.20 & 1007803.20 &           1.86 &        2.62 &                 0.77 \\
\textbf{16} & 1035440.00 & 1000209.40 &           3.40 &        3.42 &                 0.02 & 1030896.00 & 1000106.80 &           2.99 &        3.00 &                 0.01 &      1039888.60 & 1010166.80 &           2.86 &        3.84 &                 1.01 \\
\textbf{24} & 1048179.80 & 1000201.00 &           4.58 &        4.60 &                 0.02 & 1049445.80 & 1000216.20 &           4.69 &        4.71 &                 0.02 &      1054200.40 & 1008389.60 &           4.35 &        5.14 &                 0.83 \\
\textbf{28} & 1063140.80 & 1000313.60 &           5.91 &        5.94 &                 0.03 & 1054725.80 & 1000222.80 &           5.17 &        5.19 &                 0.02 &      1057040.00 & 1008122.20 &           4.63 &        5.40 &                 0.81 \\
\textbf{32} & 1071357.80 & 1000335.60 &           6.63 &        6.66 &                 0.03 & 1068250.60 & 1000250.00 &           6.37 &        6.39 &                 0.02 &      1059531.00 & 1007431.60 &           4.92 &        5.62 &                 0.74 \\
\textbf{40} & 1095432.60 & 1000462.00 &           8.67 &        8.71 &                 0.05 & 1089231.20 & 1000408.00 &           8.15 &        8.19 &                 0.04 &      1076395.00 & 1009218.60 &           6.24 &        7.10 &                 0.91 \\
\textbf{48} & 1104878.20 & 1000583.20 &           9.44 &        9.49 &                 0.06 & 1082717.20 & 1000373.40 &           7.61 &        7.64 &                 0.04 &      1086233.00 & 1010473.60 &           6.97 &        7.94 &                 1.04 \\
\textbf{56} & 1113319.60 & 1000719.00 &          10.11 &       10.18 &                 0.07 & 1079389.80 & 1000353.40 &           7.32 &        7.36 &                 0.04 &      1090696.60 & 1010237.40 &           7.38 &        8.32 &                 1.01 \\
\textbf{64} & 1104270.40 & 1000584.80 &           9.39 &        9.44 &                 0.06 & 1077770.20 & 1000360.40 &           7.18 &        7.22 &                 0.04 &      1089794.00 & 1011644.40 &           7.17 &        8.24 &                 1.15 \\
\bottomrule
\end{tabular}}
\label{difference-Torus_2D_60_directed-1000000-CHASELEV-CILK-IDEMPOTENT_LIFO}
\caption{The number of puts and takes performed during the
    spanning tree experiment on a Torus 2D 60 directed graph with an initial size
    of 1000000 items is provided. The table presents data on the
    following algorithms: Chase-Lev, Cilk THE, and
    Idempotent LIFO. Furthermore, we present the percentage difference
    between the number of puts and takes for each available thread,
    relative to the total number of puts. Finally, also we show the
    "surplus" work, which is the difference of the total number of
    \Puts (Work to be scheduled) and the total number of \Puts in
    sequential executions (i.e., 1,000,000), and the "executed surplus
    work", which is the difference between the total number of \Takes
    (actual work executed) and the total of \Takes in sequential
    executions.}
\end{table}
