\begin{table}[!ht]
\centering
\resizebox{\textwidth}{!}{\begin{tabular}{lrrrrrrrrr}
\toprule{} &    Chase-Lev &     Cilk THE &  Idempotent FIFO &  Idempotent LIFO &  Idempotent DEQUE &     WS WMult &  B. WS WMult &  WS WMult Lists &  B. WS WMult Lists \\
\midrule
\textbf{Mean               } & 206105723.49 & 211217590.84 &     222570807.61 &     187686110.36 &      311857192.28 & 107167632.11 & 952902162.60 &    223914128.86 &       375494492.90 \\
\textbf{Low Limit          } & 205713311.48 & 210753568.00 &     222029102.93 &     187305885.73 &      311337719.28 & 106944566.24 & 932470976.30 &    221872884.15 &       355430610.14 \\
\textbf{High Limit         } & 206498135.50 & 211681613.68 &     223112512.29 &     188066334.99 &      312376665.28 & 107390697.98 & 973333348.90 &    225955373.57 &       395558375.66 \\
\textbf{Confidence Interval} &    784824.01 &    928045.67 &       1083409.37 &        760449.26 &        1038946.01 &    446131.74 &  40862372.61 &      4082489.43 &        40127765.52 \\
\bottomrule
\end{tabular}}
\label{zero-pt-256-10000000-1}
\caption{The values shown in the table were calculated under
    the methodology suggested
    in~\cite{DBLP_conf_oopsla_GeorgesBE07}. These values in
    nanoseconds, are the mean time, the confidence interval limits
    (high and low) and the size region of the confidence interval. The
    zero cost experiment for puts and takes was performed with an
    initial structure size of 256 items for each worker. The
    amount of operations to perform was of 10000000 operations.}
\end{table}
