\chapter{\label{chapter:2_state_of_art}State Of The Art}

After the finish of the second world war and until the 90's, almost
all the computation was done on single-core processors and the
Operating systems simulated concurrency using schedulers and other
techniques. In 2001, IBM produced the first multi-core
processor\cite{ibmIBM100Power}, it allows two processors to work
together at very high bandwidth (for the epoch) with large on-chip
memories and high-speed buses. Since then, the number of cores in each
processor has been increasing. We must also consider Moore's law;
loosely speaking, this law tells us that each year more and more
transistors are placed into the same space (i.e., electronic
components and circuits are reduced in size), but their speed cannot
be increased without overheating. Consequently, the industry moves to
``multi-core'' architectures. In this type of architecture, multiple
processors communicate through shared memory (hardware caches, RAM),
permitting make computing more effectively using \textit{parallelism},
where the processors work together over a single
task\cite{DBLP_books_daglib_0020056}.


%%% Local Variables:
%%% mode: latex
%%% TeX-master: "../../main.tex"
%%% TeX-parse-self: t
%%% TeX-auto-save: t
%%% End:
