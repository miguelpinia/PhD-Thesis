\begin{resumen}

Concurrent computing is about the interactions between multiple computing entities over shared resources. It is considered one of the most challenging topics in computer science. Tackling the discipline requires a good imagination. We are used to thinking sequentially, and imagining multiple things happening simultaneously and randomly intermixing is not easy. This thesis explores the shift from traditional to more flexible approaches in concurrent computing for programming concurrent algorithms. It takes a theoretical approach but with practical applications in mind, particularly focusing on how relaxation can be applied to practical environments such as work-stealing and data structures (FIFO queues).

The thesis covers the state-of-the-art on classical classical concurrent computing, relaxations in concurrent computing, the problem of work-stealing, and FIFO queues. It then presents the theoretical preliminaries and the methodology used to analyze two case studies. The first case study is the problem of work-stealing. It presents two relaxed algorithms based on multiplicity and weak multiplicity based solely on read/write operations where fences are not required. The second case study delves into the problem of concurrent FIFO queues and presents a modular approach to building queue algorithms. The work concludes with an experimental evaluation of the different algorithms presented in this thesis and the conclusions and future work.


\end{resumen}
%%% Local Variables:
%%% mode: latex
%%% TeX-master: "../../thesis"
%%% End:
