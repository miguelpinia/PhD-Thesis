\begin{resumen}

Concurrent computing is about the interactions between multiple entities over shared resources. It is known that concurrent computing is one of the most challenging topics in computer science. It is not easy tackle the discipline without having a good imagination.  We are used to thinking sequentially, and it is not easy imagine multiple things happening simultaneously and randomly intermixing.

This thesis explores the shift from traditional to more flexible more flexible approaches in concurrent computing to programming concurrent algorithms. This work takes a theoretical approach but with practical applications in mind. We are interested in studying how relaxations can be applied to practical environments. In particular we are interested in applying to wrok-stealing and data-structures (FIFO queues).

In this thesis, we focus in know what is the state-of-the-art about the topics of classical concurrent computing, relaxations in concurrent computing, the problem of work-stealing and FIFO queues. After, we present the theoretical preliminaries and the methodology used to analyze the two case studies. Then, in the first case study, we analyze the problem of work-stealing and we present two relaxed algorithms based on multiplicity and weak-multiplicity. In the second case study we analyze the problem of concurrent FIFO queue, where we proposed a modular approach to build queue algorithms. We finish this work with experimental evaluation of the distinct algorithms presented in this thesis.


\end{resumen}
%%% Local Variables:
%%% mode: latex
%%% TeX-master: "../../thesis"
%%% End:
